\documentclass[12pt,a4paper]{report}
\usepackage[utf8]{inputenc}
\usepackage[frenchb]{babel}
\usepackage[T1]{fontenc}
\usepackage{graphicx}
\usepackage{lmodern}
\usepackage[hidelinks]{hyperref}
\author{Hamza Abbad}
\title{Les modes de représentation des connaissances}
\begin{document}
\maketitle
\chapter{\'Etat de l'art}
\section{Introduction}
La représentation des connaissances est un des domaines de l'intelligence artificielle. Elle consiste à modéliser
les connaissances acquisent du monde réel sous forme de concepts et de relations pour qu'elles puissent être
implémentées sous machine. Cette modélisation est souvent incomplete et non précise.\cite{WKR}

\section{Les modèles de représentation}
Depuis l'apparition de la science de l'intelligence artificielle et jusqu'à aujourd'hui, un grand nombre de modèles
de représentation de connaissances ont été proposés. Chaque représentation est utilisée avec un mode d'inférence spécifié
et elle est mieu adaptée à un problème donnée que les autres. \'Enumérer les modèles existantes est pratiquement impossible
donc nous allons discuter les types principaux de modèles qui sont :
\begin{itemize}
\item Connaissances relationnelles
\item Connaissances héritables
\item Connaissances inférentielles
\item Connaissances déclaratives
\item Connaissances procédurales
\end{itemize}
\cite{KRC}

\begin{thebibliography}{9}
\bibitem{WKR}
R. Davis, H. Shrobe, et P. Szolovits,
\emph{What is a Knowledge Representation?}.
AI Magazine, 14(1):17-33, 1993

\bibitem{KRC}
RC Chakraborty,
\emph{Knowledge Representations}.
AI Course Lecture 15 – 22, notes, slides. Juin 2010 
\end{thebibliography}
\end{document}