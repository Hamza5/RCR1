\documentclass[12pt,a4paper]{report}
\usepackage[utf8]{inputenc}
\usepackage[frenchb]{babel}
\usepackage[T1]{fontenc}
\usepackage{graphicx}
\usepackage{lmodern}
\usepackage[hidelinks]{hyperref}
\usepackage{amsthm}
\author{Hamza Abbad}
\title{Les modes de représentation des connaissances}
\newtheorem{example}{Exemple}
\begin{document}
\maketitle
\chapter{\'Etat de l'art}
\section{Introduction}
La représentation des connaissances est un des domaines de l'intelligence artificielle. Elle consiste à modéliser
les connaissances acquisent du monde réel sous forme de concepts et de relations pour qu'elles puissent être
implémentées sous machine. Cette modélisation est souvent incomplete et non précise.\cite{WKR}

\section{Les modèles de représentation}
Depuis l'apparition de la science de l'intelligence artificielle et jusqu'à aujourd'hui, un grand nombre de modèles
de représentation de connaissances ont été proposés. Chaque représentation est utilisée avec un mode d'inférence spécifié
et elle est mieu adaptée à un problème donnée que les autres. \'Enumérer les modèles existantes est pratiquement impossible
donc nous allons discuter les types principaux de modèles qui sont :
\begin{itemize}
\item Connaissances relationnelles
\item Connaissances héritables
\item Connaissances inférentielles
\item Connaissances déclaratives
\item Connaissances procédurales\cite{KRC}
\end{itemize}

\chapter{Modèles de représentation des connaissances}
\section{Modèle de connaissances relationnelles}
Ce type de représentation associe les éléments d'un domaine avec ceux de l'autre en utilisant les attributs d'un objet et
ses valeurs. Les réalités sont stockés sous forme de correspondances des éléments entre les domaines. Ce modèle n'est pas
généralement efficace pour l'inférence.\cite{KRC}
\begin{example}
Un exemple simple d'une représentation relationnelle

\begin{table}[h]
\centering
\begin{tabular}{|c|c|c|}
\hline
Utilisateur & e-mail & domaine \\
\hline
Fayçal & lan@example.com & RT \\
\hline
Sohayb & cmpl@example.com & IA \\
\hline
Fadi & fnct@example.com & DL \\
\hline
Selman & exinf@example.com & IA \\
\hline
Sofian & wlan@example.com & RT \\
\hline
\end{tabular}
\end{table}
En utilisant cette représentation on peut répondre directement à la question "Qui sont les utilisateurs du domaine IA ?"
mais pas la question "Quel est le domaine qui a le moins d'utilisateurs ?" (sauf si on définit une procédure
pour faire cela).
\end{example}

\section{Modèles de connaissances héritables}
Dans ce type les éléments de connaissances héritent leurs attributs à partir de leurs parents. Les éléments fils
peuvent définir leurs propres attributs et redéfinir ceux de leurs parents. L'héritage est une forme importante
d'inférence mais pas suffisante seule, il faut qu'elle soit renforcé par d'autres mécanismes d'inférence.
Une des représentations des connaissances héritables les plus utilisés est \emph{les réseaux sémantiques}.\cite{KRC}

\begin{example}
Ci-dessous est une représentation en utilisant les réseaux sémantiques:

\includegraphics[scale=0.55]{pictures/reseau_semantique}

Les noeuds représentent les entitées de la connaissance et les relations entre ces noeuds sont les attributs.
Il y a des relations spéciales comme \texttt{est un} représentant l'héritage et \texttt{instance} qui
représente que le noeud fils est un objet de la classe représentée par le noeud parent.
\end{example}

\section{Modèles de connaissances inférentielles}
Dans ce type de représentation des nouvelles connaissances sont générées à partir de celles existantes.
Les connaissances générées ne dépendent plus des connaissances originales pour qu'elles soient valides, mais
il se peut qu'elles les nécessitent pour plus d'inférence.\cite{KRC} La logique propositionnelle et celle de
prédicats utilisent ce type de connaissances.

\begin{example}
Nous avons les propositions suivantes :
\begin{enumerate}
\item \textbf{U(x)} : \textbf{x} est un Ultrabook
\item \textbf{E(x)} : Le diamètre de l'écran de \textbf{x} est au plus 13.3".
\item \textbf{P(x)} : Les performances de \textbf{x} sont élevés.
\item \textbf{S(x)} : $ E(x) \wedge P(x) \rightarrow U(x) $
\item \textbf{D} : Le diamètre d'écran de Dell XPS 13 est 13.3".
\item \textbf{C} : Les performances de Dell XPS 13 sont élevés.
\end{enumerate}
A partir des propositions précédentes on peut déduire la conclusion suivante :
\textbf{L} : "Dell XPS 13 est un Ultrabook"  en utilisant la logique des prédicats comme suit :
De \textbf{D} on sait que \textbf{E(Dell XPS 13)} = vrai, et à partir de \textbf{C} on a \textbf{P(Dell XPS 13)} = vrai
alors à partir de ces propositions et de \textbf{S(x)} on peut déduire que \textbf{U(Dell XPS 13)} = vrai
\end{example}

\begin{thebibliography}{9}
\bibitem{WKR}
R. Davis, H. Shrobe, et P. Szolovits,
\emph{What is a Knowledge Representation?}.
AI Magazine, 14(1):17-33, 1993

\bibitem{KRC}
RC Chakraborty,
\emph{Knowledge Representations}.
AI Course Lecture 15 – 22, notes, slides. Juin 2010 
\end{thebibliography}
\end{document}