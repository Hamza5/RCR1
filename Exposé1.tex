\documentclass[12pt,a4paper]{report}
\usepackage[utf8]{inputenc}
\usepackage[frenchb]{babel}
\usepackage[T1]{fontenc}
\usepackage{graphicx}
\usepackage{lmodern}
\usepackage[hidelinks]{hyperref}
\usepackage{amsthm}
\author{Hamza Abbad}
\title{Les modes de représentation des connaissances}
\newtheorem{example}{Exemple}
\begin{document}
\maketitle
\chapter{\'Etat de l'art}
\section{Introduction}
La représentation des connaissances est un des domaines de l'intelligence artificielle. Elle consiste à modéliser
les connaissances acquisent du monde réel sous forme de concepts et de relations pour qu'elles puissent être
implémentées sous machine. Cette modélisation est souvent incomplete et non précise.\cite{WKR}

\section{Les modèles de représentation}
Depuis l'apparition de la science de l'intelligence artificielle et jusqu'à aujourd'hui, un grand nombre de modèles
de représentation de connaissances ont été proposés. Chaque représentation est utilisée avec un mode d'inférence spécifié
et elle est mieu adaptée à un problème donnée que les autres. \'Enumérer les modèles existantes est pratiquement impossible
donc nous allons discuter les types principaux de modèles qui sont :
\begin{itemize}
\item Connaissances relationnelles
\item Connaissances héritables
\item Connaissances inférentielles
\item Connaissances déclaratives
\item Connaissances procédurales\cite{KRC}
\end{itemize}

\chapter{Modèles de représentation des connaissances}
\section{Modèle de connaissances relationnelles}
Ce type de représentation associe les éléments d'un domaine avec ceux de l'autre en utilisant les attributs d'un objet et
ses valeurs. Les réalités sont stockés sous forme de correspondances des éléments entre les domaines. Ce modèle n'est pas
généralement efficace pour l'inférence.\cite{KRC}
\begin{example}
Un exemple simple d'une représentation relationnelle

\begin{table}[h]
\centering
\begin{tabular}{|c|c|c|}
\hline
Utilisateur & e-mail & domaine \\
\hline
Fayçal & lan@example.com & RT \\
\hline
Sohayb & cmpl@example.com & IA \\
\hline
Fadi & fnct@example.com & DL \\
\hline
Selman & exinf@example.com & IA \\
\hline
Sofian & wlan@example.com & RT \\
\hline
\end{tabular}
\end{table}
En utilisant cette représentation on peut répondre directement à la question "Qui sont les utilisateurs du domaine IA ?"
mais pas la question "Quel est le domaine qui a le moins d'utilisateurs ?" (sauf si on définit une procédure
pour faire cela).
\end{example}


\begin{thebibliography}{9}
\bibitem{WKR}
R. Davis, H. Shrobe, et P. Szolovits,
\emph{What is a Knowledge Representation?}.
AI Magazine, 14(1):17-33, 1993

\bibitem{KRC}
RC Chakraborty,
\emph{Knowledge Representations}.
AI Course Lecture 15 – 22, notes, slides. Juin 2010 
\end{thebibliography}
\end{document}